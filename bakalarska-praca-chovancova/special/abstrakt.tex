%--------------------------------------------------------------------------------------
%%% slovensky abstrakt

\begin{abstract}

\noindent
{\sc Chovancová Oľga:} {\em Vizualizácia dát získaných pomocou SCADA systémov s využitím HTML 5 štandartov}
[Bakalárska práca] 

\noindent
Žilinská Univerzita v~Žiline,  
Fakulta riadenia a informatiky,  
Katedra softvérových technológií.

\noindent  
Vedúci: Ing. Juraj Veverka 
 
\noindent
Tútor	Ing. Patrik Hrkút, PhD.

\noindent  
Stupeň odbornej kvalifikácie:
Bakalár Informatiky
%Inžinier v študijnom odbore Telekomunikačné siete, na Elektrotechnickej fakulte na Žilinskej univerzite v Žiline. 


\bigskip
Téma práce je vizualizácia dát získaných pomocou SCADA systémov. Cieľ práce je nájsť postup tvorby a vizualizácie grafických komponentov. Produktom bakalárskej práce je grafický komponent na vizualizáciu technologických procesov s využitím HTML 5  štandardov.  V práci je popísaný detailný postup vizualizácie prečerpávacej stanice. Bol navrhnutý jednoduchý interface, pomocou ktorého komponenty komunikujú so serverovou časťou SCADA systému. 
Cieľová platforma pre výslednú webovú aplikáciu bude kompatibilná s rodinou štandardov HTML 5 pre každý webový prehliadač. 

\noindent
Kľúčové slová: vizualizácia, animácie, HTML5, JavaScript, Snap.svg.js, SCADA, REST API

\end{abstract}


%--------------------------------------------------------------------------------------
%%% anglicky abstrakt


\selectlanguage{english}
\begin{abstract}

\noindent
{\sc Chovancová Oľga:} {\em Data visualization acquired by SCADA systems using HTML5 standarts}
[Bacalar thesis] 

\noindent
University of Žilina,  
Faculty of Management Science and Informatics, 
Department of 
 
\noindent
Tutor:  Ing. Juraj Veverka.\\
Tutor:  Ing. Patrik Hrkú
t, PhD.
 
\noindent
Qualification level: 

%
%Engineer in field University of Zilina, Faculty of Telecommunication, Fixed networks.
%Solution Design Architect %- member of team who develops system D2000. 

\noindent
 TODO

\bigskip


The main idea of this is the visualization of data obtained by SCADA systems. Aim is to provide a process of creation and visualization of graphical components. Product of the thesis is a graphical component for visualization of technological processes using HTML 5 standards. The paper describes the detailed process visualization of pumping station. It was designed simple interface through which components communicate with the server part of the SCADA system. The target platform for the resulting web application will be compatible with your family the HTML 5 for each Web browser.


\end{abstract}
\selectlanguage{slovak}