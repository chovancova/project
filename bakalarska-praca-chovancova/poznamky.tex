{\Huge\textbf{ DELETE THIS PAGE}}
\section{Poznámky - informačný účel}
Táto strana slúži na poznámky, TODO, toho čo mám ešte urobiť. 
Bude samozrejme zmazaná.  
Dany text, ktory sa bude dat pouzit v praci bude presunuti do jadra prace. 

dopnit do praktickej casti mapu: vyuzivajuc tento priklad ovladania.. \url{http://raphaeljsvectorgraphics.com/the-graphical-web/turtle-graphics-logo}
pozriet si funkciu snap - inAnim()...


Pridat postup ako sa ziska Snap zo stranky a implementuje do html dokumentu. .

ujednotnit nazov attribut a parameter

dalsie co ma napadlo - je kapitola \cite[p.~81]{Dawber} ktora je o praci s existujucimi svg subormi / kedze to takmer na konci, tak by som mohla prehodit taktiez kapitoly, ze najprv budem pisat o svg knizniciach, ako sa to tam tvori, a tak dalej, a ptom prejdem na inkscape a priklad a animaciu .. 


Making our SVG responsive

There are 2 key points to making our svg responsive. The width/height must be 100\% and a viewBox MUST be defined.

Výška a šírka musí byť udaná relatívne, napríklad v percentách. Musí byť nastavený aj viewBox. 
To môžem realizovať dvoma spôsobmi:
1. sposob
var s = Snap("#svgout"); 
s.attr({ viewBox: "0 0 600 600" });
2. sposob pri definicii svg $ <svg id="idsvg" viewBox="0 0 600 600" weight="100\%" height="100\%"> $


\section{Kapitoly by mali byt nasledovne}
\begin{enumerate}
	\item ciel prace 
	\item metodika prace
	\item definovanie zakladnych pojmov / co je html5, co je scada system, a co je svg vektorova grafika (takmer hotove, len nemam scada systemy definovane), 

	\item ANALYZA javascriptovych kniznic / (toto uz je takmer hotove, este tam pridat jquery )
	\item Postup \textbf{tvorby} grafickych komponentov cez Snap...
		\item analyza nastrojov - preco som sa rozhodla pouzit inkscape na tvorbu svg obrazkov, //este tam musim spomenut moznost exportu do formatov... 
		
		\item postup na vytvorenie OBRAZKA SVG V INKSCAPE a integracia do snapu  \cite[p.~82-5]{Dawber} 
		\item priklad vytvorenia precepavacej stanice v Inkscape (toto je uz takmer hotove) 
	\item Postup \textbf{animacie} uz vytvorenych grafickych elementov 
	\item Priklad v kode animovanie Precerpavaciej stanice... (Takmer hotove) toto je vlastne ta implementacia vzorovej sady grafickych komponentov
	\item Navrh REST API na prepojenie grafickych komponentov so SCADA SERVEROM  \textit{toto no nemam / mam iba priklad kodu jednoducheho }
	\item Analyza moznosti automatickeho mapovania api grafickych prvkov pomocou metadat na existujuce api dostupne pre scada serverr D2000 \textit{tooto nemam vobec}
	\item analyza vykonnosti a vykonnostne obmedzenia - toto iba okrajovo spomeniem - vseobecne preco je lepsie svg, a preco nie je vhodne, v prvej kapitole som pisala rozdiel medzi canvas a svg - istym sposobom to je analyza obmedzeni.. ?
	\item zhrnutie
	
\end{enumerate}
