\chapter{REST API}

\section{REST API pre grafické komponenty}
Grafické komponenty pre vizualizáciu dát zo systému D2000 budú umiestňované na HTML stránkach použitých ako súčasť web rozhrania frameworku D2000 WebSuite.

Tento framework je založený na technológii Java Enterprise Edition a Java Server Faces.

Životný cyklus webovej stránky - načítanie stránky s technologickou schémou, pre prvé zobrazenie kompletnej schémy je potrebný plný dáta set.


Príklad REST URL:\\
\url{http://localhost:8080/scada-demo/rest/pumpingstation/gatfulldataset}


Príklad JSON dát z grafického komponentu prečerpávacia stanica: 
\begin{lstlisting}
var updateData = {
	"valve1": "false",
	"tank": "0",
	"engineRotation": "false",
	"valve2": "false"
};
\end{lstlisting}


Zobrazenie stránky v rámci jednej http session, typicky vo web aplikáciách kde sa užívateľ prihlási pomocou mena a hesla, trvanie jeho session je obmedzené na predom stanovený čas, napríklad jednu hodinu.


Pri zobrazení zložitejšej technologickej schémy, je potrebné optimalizovať množstvo prenesených dát a interakciu z DOM stránky. Preto je počas zobrazenia schémy výhodne implementovať čiastočne aktualizácie, ktoré menia len dotknuté časti schémy a nie celú schému ako je tomu pri načítaní stránky.  

Príklad REST URL:\\ 
\url{http://localhost:8080/scada-demo/rest/pumpingstation/getvalvestatus}
príklad JSON dát:
\begin{lstlisting}
var updateDataTank = {"tank": "86"};
\end{lstlisting}

príklad REST URL\\ \url{http://localhost:8080/scada-demo/rest/pumpingstation/getrotorstatus}
príklad JSON dát: %TODO(môžeš uviesť dáta z puding stati on)
príklad REST URL\\ \url{http://localhost:8080/scada-demo/rest/pumpingstation/getwaterlevel}
príklad JSON dát: %TODO(môžeš uviesť dáta z puding stati on)


\section{Data binding pre system D2000}
Priama väzba v prípade jednoduchých schém. Jednoduchá schéma predstavuje vizualizáciu meraného alebo počítaného bodu v systéme D2000. Takáto vizualizácia je realizovaná pomocou widgetu jedna sa o mapovanie 1:1.
%(môžeš uviesť príklad ako teplomer alebo ručičkový merač, obrázky) 
Zložitejšie schémy pozostávajúce z vizualizácie výrobného procesu alebo komplexnej technológie sú mapované vo vzťahu n:1. Jedna schéma vizualizuje dáta z n meraných alebo počítaných bodov, pripadne získava dáta cez asynchrónne volania \ac{RPC} systému D2000.

Ak to vyžaduje logika aplikácie, server implementuje dátový zdroj (service), ktorý agreguje dáta a systémové udalosti. Pre dosiahnutie real-time odozvy web rozhrania výhodne použiť obojstrannú komunikáciu medzi web browserom a serverom - technológiu web sockets.

% (môžeš uviesť niečo o web socketoch, skús pohľadať na webe), vo stvrtok to mozme doladit.