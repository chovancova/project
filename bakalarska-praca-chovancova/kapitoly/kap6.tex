\chapter{Návrh \acs{REST} \acs{API}}

%JSON

%23 strana knihy restful web apis

\section{\acs{JSON}}


%%
%\ac{JSON} predstavuje spôsob, ako poskytovať objekty JavaScriptom ako dáta namiesto kódovania týchto dát do dokumentu XML. ....TODO

TODO PREROBIT CELE - TODO - 
%Formát \ac{JSON} je založený na rovnakom princípe ako sa tvoria objekty v JavaScripte. Dátový formát je schopný reprezentovať rovnaké typy dát ako jazyk \acs{XML}. \cite[p.~622-4]{Zakas}

%JSON je efektívnejší spôsob ako posielať dáta od serveru klientovi, pretože nie je potrebné spracovať odpoveď pomocou \acs{DOM} a dáta je možné použiť ihneď, bez nutnosti explicitnej konverzie na objekty JavaScriptu. \cite[p.~280]{Suehring}







%\subsection{JSON Syntax}




%Napríklad ko  
%
%Pravidlá pre tvorbu JSON:
%
%\begin{itemize}
%	\item Dáta sú pároch - meno/hodnota 
%	\item Dáta musia byť v úvodzovkách, napríklad "fillPerc":"50"
%	\item Dáta sú oddelené čiarkou
%	\item Kučeravé zátvorky uchovávajú objekty
%	\item Hranaté zátvorky ukladajú pole
%	\item Nemôže obsahovať komentáre
%	\item 
%\end{itemize}

%štandard - pre JSON
%http://www.ecma-international.org/publications/files/ECMA-ST/Ecma-262.pdf


API k Pumping station schéme. 

\begin{lstlisting}
var updateData = {
	"valve": "true",
	"tank": "20",
	"engine": "20"
};
\end{lstlisting}

Tento kód definuje objekt s názvom updateData, ktorá má tri vlastnosti. 




 Interface funkcia k REST API. 
 
 \begin{lstlisting}
 function updateSchema(updateData){
	 updateSchema01(updateData.valve, updateData.tank, updateData.engine);
 }
 \end{lstlisting}
