%\chapter{Automatické mapovanie API}
%
%Analyzujte možnosti automatického mapovania \acs{API} grafických prvkov pomocou metadát na existujúce API dostupné pre \acs{SCADA} server D2000.
%

\section{REST API}


\section{REST API pre graficke komponenty}
Graficke koponenty pre vizualizaciu dat zo systemu D2000 budu umiestnovane na html strankach pouzitych ako sucast web rozhrania frameworku D2000 WebSuite.
Tento framework je zalozeny na technologii Java Enterprise Edition a Java Server Faces.
Zivotny cyklus web stranky - nacitanie stranky s technologickou schemou, pre prve zobrazenie kompletnej schemy je potrebny plny data set.
priklad REST URL\\ http://localhost:8080/scada-demo/rest/pumpingstation/gatfulldataset
priklad JSON dat: (mozes uviest data z puping station)
Zobrazenie stranky v ramci jednej http session, typicky vo web aplikaciach kde sa uzivatel prihlasi pomocou mena a hesla, trvanie jeho session je obmendzene na predom stanoveny cas, napriklad jednu hodinu.
Pri zobrazeni zlozitejsej technologickej schemy, je potrebne optimalizovat mnozstvo prenesenych dat a interakciu z DOM stranky. Preto je pocas zobrazenia schemy vyhodne implementovat ciastocne aktualizacie, ktore
menia len dotknute casti schemy a nie celu schemu ako je tomu pri nacitani stranky.  
priklad REST URL\\ http://localhost:8080/scada-demo/rest/pumpingstation/getvalvestatus
priklad JSON dat: (mozes uviest data z puping station)
priklad REST URL\\ http://localhost:8080/scada-demo/rest/pumpingstation/getrotorstatus
priklad JSON dat: (mozes uviest data z puping station)
priklad REST URL\\ http://localhost:8080/scada-demo/rest/pumpingstation/getwaterlevel
priklad JSON dat: (mozes uviest data z puping station)


\section{Data binding pre system D2000}
Priama vazba v pripade jednoduchych schem. Jednoducha schema predstavuje vizualizaciu meraneho alebo pocitaneho bodu v systeme D2000. Takato vizualizacia je realizovana pomocou widgetu
(mozes uviest priklad ako teplomer alebo rucickovy merac, obrazky) jedna sa o mapovanie 1:1.
Zlozitejsie schemy pozostavajuce z vizualizacie vyrobneho procesu alebo komplexnej technologie su mapovane vo vztahu n:1, teda jedna schena vizualizuje data z n meranych alebo pocitanych bodov,
pripadne ziskava data cez asynchronne volania RPC (remote procedure call) systemu D2000.
Ak to vyzaduje logika aplikacie, server implementuje datovy zdroj (service), ktory agreguje data a systemove udalosti. Pre dosiahnutie real-time odozvy web rozhrania vyhodne pouzit obojstrannu
komunikaciu medzi web browserom a serverom - technologiu web sockets. (mozes uviest nieco o web socketoch, skus pohladat na webe), vo stvrtok to mozme doladit.



