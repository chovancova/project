% !TeX root = ../main.tex
% !TeX spellcheck = sk_SK
% !TeX encoding = UTF-8
\chapter{Základné pojmy}
V kapitole sú popísané základné pojmy. 

\section{\acs{HTML}5 štandard}

\ac{W3C} vydalo štandard \acs{HTML}5 dňa 28. októbra 2014. 
HTML5 je podporovaný vo všetkých moderných webových prehliadačoch. 
Na obrázku \ref{fig:obrazokHTML} je HTML5 \acs{API} a súvisiace taxonómia technológií a ich status. 

\begin{center}
	\begin{figure}[hp]
\centering
\includegraphics[width=0.7\linewidth]{obrazky/obrazokHTML}
\caption{HTML 5 API}
\label{fig:obrazokHTML}
\end{figure}
\end{center}

HTML5 Graphics definuje dva spôsoby vykreslenia využívajúc: 
\begin{itemize}
	\item $<$canvas$>$ - JavaScript
	\item $<$svg$>$ - SVG
\end{itemize}


\section{Čo je SVG?}
\ac{SVG} je štandardný formát pre vektorovú grafiku. Vektorová grafika je definovaná cez body, priamky, mnohouholníky, elipsy, krivky, alebo iné geometrické tvary.  

\acs{SVG} je jazyk na opísanie dvojrozmernej grafiky v   \ac*{XML}. Vďaka tomu, umožňuje reprezentáciu grafických informácii v kompaktnom a prenositeľnom tvare.

 SVG povoľuje tieto tri typy grafických objektov: vektorové grafické tvary, obrázky a text. 
Grafické objekty môžu byť zoskupené, štylizované, zmenené, a kombinované do predošlých vrstiev objektov. 

SVG obrázky môžu byť dynamické a interaktívne.

Prispôsobiteľnosť SVG umožňuje zmeniť veľkosť grafického komponentu bez straty kvality vzhľadu. Čo umožňuje zobraziť responzívne na viacerých možných zariadení. 
SVG sa bude zobrazovať rovnako na rôznych platformách. Je kompatibilná s štandardmi \acs{HTML}5, ktoré navrhla \ac*{W3C}. 


 \subsection{Podpora v webovom prehliadači}
 Súčasné prehliadače plne podporujú $<$svg$>$ elementy.  
  Čísla v tabuľke \ref{svgpreh} špecifikujú prvé verzie webových prehliadačov, ktoré sú schopné zobraziť $<$svg$>$ element.\cite{w3svg}
  
\begin{table}[hp]
\begin{center}
		\begin{tabular}{|c|c|c|c|c|c|}
		\hline \textbf{Element} & \textbf{Chrome} & \textbf{Internet} \textbf{Explorer}  & \textbf{Firefox}  & \textbf{Safari} & \textbf{Opera}  \\ 
		\hline $<svg>$ & 4.0& 9.0 & 3.0 & 3.2  &   10.1 \\ 
		\hline 
	\end{tabular} 
\end{center}
	
	\caption{Podpora HTML $<svg>$ elementu v webových prehliadačoch}
	\label{svgpreh}
\end{table}
 
 \begin{figure}
\centering
\includegraphics[width=0.7\linewidth]{obrazky/podpora}
\caption{Podpora SVG vo webových prehliadačoch}
\label{fig:podpora}
\end{figure}
% http://caniuse.com/#feat=svg
 
 \subsection{Rozdiely medzi SVG a Canvas}
TODO \url{http://www.petrpexa.cz/diplomky/trantyr.pdf} strana 52

SVG patrí do vektorovej grafiky a Canvas zase do raster bitmap grafiky. 
 SVG je jazyk na opísanie dvojrozmernej grafiky v XML.  Canvas kreslí dvojrozmernú grafiku za behu programu cez JavaScript.  SVG je XML založený, čo znamená, že každý element je dostupný cez SVG DOM.   JavaScript umožňuje ovládanie udalostí elementov. V SVG je každý tvar zapamätaný ako objekt.  V prípade zmeny $<$svg$>$ elementu sa automaticky prekreslí.  
 
 
 Canvas je prekresľovaný pixel za pixelom. Prehliadač na neho zabudne, ako náhle sa vykreslí. Keď chcem zmeniť jeho pozíciu, musím prekresliť úplne všetko. 
 
 
 %SVG is XML based, which means that every element is available within the SVG DOM. You can attach JavaScript event handlers for an element.
 %In SVG, each drawn shape is remembered as an object. If attributes of an SVG object are changed, the browser can automatically re-render the shape.
 %Canvas is rendered pixel by pixel. In canvas, once the graphic is drawn, it is forgotten by the browser. If its position should be changed, the entire scene needs to be redrawn, including any objects that might have been covered by the graphic.
 
 
 \subsection{Porovnanie Canvas a SVG}
 Tabuľka \ref{canvas:SVG} zobrazuje niekoľko dôležitých odlišností medzi Canvas a SVG. 
% The table below shows some important differences between Canvas and SVG:
 TODO DPI
 \begin{table}[hp]
 \centering
 \begin{tabular}{|l|p{7.5cm} |}
 	\hline \textbf{Canvas} & \textbf{SVG} \\
 	 	\hline Závislé na rozlíšení a \acs{DPI} & Nezávislé na rozlíšení a DPI \\ 
 	\hline Nepodporuje dynamické zmeny & Podporuje dynamické zmeny \\ 
 	\hline Obmedzené možnosti na vykresľovanie  & Vhodné pre aplikácie s veľkými plochami na vykresľovanie \\ 
 	\hline & Väčší výpočtový výkon pri komplexnom obrázku \\ 
 	\hline Vhodné pre grafické-intenzívne hry & Nevhodné pre dynamické hry \\ 
 	\hline 
 \end{tabular} 

 \caption{Porovnanie Canvas a SVG}
 \label{canvas:SVG}
 
\end{table}
 
 
 \section{Základná syntax \acs*{SVG}}

V HTML5 sa môžu používať vložené SVG elementy priamo v na HTML stránke. 
%https://css-tricks.com/using-svg/



SVG can be created 
-   inline: within the HTML document 
-   by embedding a stand alone .SVG file


Copy/paste SVG code within HTML code (inlining)
Using the HTML img tag
Using the HTML object tag
Using the HTML iframe tag
Using CSS (background images)
Including SVG within SVG using the image tag.


\begin{table}[hp]
	\begin{center}
		\begin{tabular}{|l|l|}
			\hline \textbf{Technika} & \textbf{Popis} \\ 
			\hline $<$embed$>$ tag & Načíta vytvorený SVG súbor.  \\ 
			\hline $<$object$>$ tag & Nepovoľuje skriptovanie.  \\ 
			\hline $<$iframe$>$ tag & Zobrazí SVG v rámci  \\ 
			\hline Inline & Vytvorí Svg súbor \\ 
			\hline 
		\end{tabular} 
	\end{center}
	\caption{Spôsoby vytvorenia SVG v HTML dokumente}
	\label{vytvorenie:SVG}
\end{table}

Príklady načítania SVG v HTML dokumente.

$<$img$>$ tag 

\begin{lstlisting}
	Image:
	<img src="stanica2.svg" width = "50" height= "50" />
	
	Embed:
	<embed src="stanica2.svg" width = "50" height= "50" />
	
	Object:
	<object type="image/svg+xml" data="stanica2.svg"
	width="50" height="50"></object>
	
	Iframe:
	<iframe src="stanica2.svg" width = "50" height= "50"><</iframe>
	
\end{lstlisting}






\subsection{Príklad použitia SVG v HTML dokumentu s inline SVG }

HTML kód: 

\begin{lstlisting}
<!DOCTYPE html>
<html>
<head lang="sk">
<meta charset="UTF-8">
<title></title>
</head>
<body>

	<svg width="100" height="100">
		<circle cx="50" cy="50" r="40" stroke="black" stroke-width="2" fill="silver" />
	</svg>	
	
</body>
</html>

\end{lstlisting}




SVG obrázok začína s $<$svg$>$ elementom. Atribúty elementu $<$svg$>$ sú width a height. Definujú šírku a výšku SVG obrázka. Element $<$circle$>$ je použitý na nakreslenie kruhu. Atribúty cx, cy definujú x, y súradnice od centra kruhu. Ak je cx, cy vynechané, tak center kruhu je nastavený na $($0, 0$)$. Atribút r  definuje polomer kruhu. Atribúty stroke a stroke-width určujú to ako bude vyzerať obrys útvaru. Kruh má nastavený 2px čierny okraj. 
Atribút fill vyplní vnútro kruhu. V príklade je vyplnený sivou farbou. Tag, ktorý uzavrie SVG obrázok je $<$$/$svg$>$. Keďže SVG je validné XML, tak všetky elementy musia byť správne zatvorené. 
%zdroj www.w3schools.com/svg/svg_inhtml.asp


Vykreslí na HTML webovú stránku útvar, ktorý je na obrázku \ref{jednoduchyKruh}.

\begin{figure}[hp]
	\begin{center}
		\includegraphics  {obrazky/jednoduchyKruh.png}
		\caption{Vykreslenie SVG na HTML stránke}
		\label{jednoduchyKruh}
	\end{center}
\end{figure}


\section{SVG útvary} 

\acs*{SVG} má preddefinované tieto tvary elementov:
\begin{itemize}
	\item Obdĺžník $<$rect$>$
	\item Kruh $<$circle$>$
	\item Elipsa $<$ellipse$>$
	\item Čiara $<$line$>$
	\item Polyline $<$polyline$>$
	\item Mnohouholník $<$polygon$>$
	\item Cesta $<$path$>$	
\end{itemize}

TODO ESTE NIECO PRIDAT

%\section{CSS vlastnosti - tuto kapitolu asi vyhodim}
%
%Podľa HTML5 štandardov dokáže CSS meniť vlastnosti SVG.
%
%%www.w3.org/tr/svg/styling.html
%%http://www.w3.org/TR/SVG/propidx.html http://www.w3.org/TR/SVG2/styling.html
%
%Vymenované vlastnosti, ktoré majú rovnaké \acs{CSS} 2.1 a \acs{SVG}. 
%
%Vlastnosti písma: 
%‘font’,
%‘font-family’,
%‘font-size’,
%‘font-size-adjust’,
%‘font-stretch’,
%‘font-style’,
%‘font-variant’,
%‘font-weight’.
%
%Vlastnosti textu: 
%‘direction’,
%‘letter-spacing’,
%‘text-decoration’,
%‘unicode-bidi’,
%‘word-spacing’.
%
%Ďalšie vlastností:
%‘clip’,
%‘color’ (‘fill’, ‘stroke’, ‘stop-color’, ‘flood-color’ a ‘lighting-color’), 
%‘cursor’,
%‘display’,
%‘overflow’,
%‘visibility’.
%
%Nasledujúce vlastností nie sú definované v CSS 2.1. 
%
%Clipping, Masking and Compositing properties:
%‘clip-path’, 
%‘clip-rule’,
%‘mask’, 
%‘opacity’.
%
%
%Filter Effects properties:
%‘enable-background’
%‘filter’
% ‘flood-color’
% ‘flood-opacity’
% ‘lighting-color’	
%
%
%Gradient properties: ‘stop-color’, 	‘stop-opacity’. 
%
%Interactivity properties:
%\begin{itemize}
%\item 	‘pointer-events’
%\end{itemize}
%Color and Painting properties:
%\begin{itemize}
%\item ‘color-interpolation’, ‘color-rendering’
%\item ‘fill’, ‘fill-opacity’, ‘fill-rule’
%\item ‘image-rendering’
%\item ‘marker’, ‘marker-end’, ‘marker-mid’,‘marker-start’
%\item ‘shape-rendering’
%\item ‘stroke’, ‘stroke-dasharray’, ‘stroke-dashoffset’, ‘stroke-linecap’, ‘stroke-linejoin’, ‘stroke-miterlimit’, ‘stroke-opacity’, ‘stroke-width’
%\item ‘text-rendering’	
%\end{itemize}
%
%Text properties:
%\begin{itemize}
%\item 	‘alignment-baseline’
%\item 	‘baseline-shift’
%\item 	‘dominant-baseline’
%\item 	‘glyph-orientation-horizontal’
%\item 	‘glyph-orientation-vertical’
%\item 	‘text-anchor’
%\item 	‘writing-mode’
%\end{itemize}
%
% 
%
%mozno tu este pridam nieco z prezentacii, ktore ma ucitel mikus
