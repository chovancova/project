%\chapter{Implementácia komponentov}
\chapter{Návod na prácu s Snap.svg.js}

\section{HTML súbor}
Do HTML súboru index.html pridáme párový tag $<$svg$>$. Na toto miesto sa neskôr vykreslí SVG načítané zo súboru cez JavaScript. Môže sa tu uviesť i celý kód SVG obrázka. V prípade, že nebude v dokumente dané kde presne sa nachádza SVG tag tak sa pridá na najbližšie voľné miesto. 
\subsection{Kód}
\begin{lstlisting}
<svg 
	id="svgStanica" 
	viewBox="0 0 750 600" 
	width="40%" 
	height="40%"> 
</svg>
\end{lstlisting}

\subsection{Vysvetlenie kódu}
\begin{itemize}
\item  \textbf{id} - jedinečný identifikátor, cez ktorý meníme vlastnosti.
\item 	\textbf{viewBox} - je virtuálne okno, ktorým sa užívateľ uvidí svg obrázok. Je atribút, ktorý povoľuje špecifikovať danú množinu grafických komponentov, aby sa zobrazili v daných súradniciach x, y a šírke, výške. Hodnoty atribútov v viewBox sú štyri čísla - min-x, min-y, width a height. 
\item 	\textbf{width} a \textbf{height} je šírka a výška. Hodnoty atribútov je možné uviesť relatívne v percentách, alebo absolútne v pixloch. 
\end{itemize}

Musíme sa uistiť, aby sa načítali všetky JavaScriptové knižnice, pred spustením funkcií. To zabezpečíme pridaním  onload do tagu $<$body$>$. 
\begin{lstlisting}
	<body onload="onPageLoad();">
\end{lstlisting}

A ešte jedna vec pri HTML súbore - TODO. 
\begin{lstlisting}
  <script type="text/javascript" src="../js/snap.svg-min.js"></script>
  <script type="text/javascript" src="PumpingStation.js"></script>
\end{lstlisting}


\section{PumpingStation.js}

V súbore PumpingStation.js sú funkcie na animovanie.. TODO
\subsection{onPageLoad()}
Táto funkcia sa spustí pri načítaní tela HTML súboru index.html. Funkcia spustí funkciu PumpingStation(). Prvý parameter je udaný konkrétny svg súbor, ktorý chcem načítať. Druhý parameter je id tagu svg, ktorý je v html. TODO

\begin{lstlisting}
function onPageLoad() {
	PumpingStation("PumpingStation.svg", "#svgStanica" );
}
\end{lstlisting}

\subsection{PumpingStation(par1, par2)}

Funkcia inicializuje daný svg súbor, a vykreslí ho. 
Parametre pre PumpingStation je názov svg súboru, a id, ktoré sa nachádza v tagu $<$svg$>$ html súbore.

\begin{lstlisting}
var PumpingStation = function(nazovFileSVG, nameHTMLidSVG) {
	paper = Snap(nameHTMLidSVG);
	Snap.load(nazovFileSVG, function (f) {
		paper.append(f);
	});
};
\end{lstlisting}

\textbf{paper} - bude globálna premenná. Vytvorí plochu na  kreslenie, alebo  wraps existujúci SVG element. Ako parametre môžu byť buď šírka, výška, alebo DOM element. Napríklad Snap(600, 800), alebo Snap("\#svgStanica"), resp Snap(). 

Pomocou load načítam vytvorený svg súbor. Na plochu ho zobrazím pomocou príkazu append. 

\subsection{Tank}
Zanimovanie stupania a klesania hladiny nadrze. 
\begin{lstlisting}
var Tank = {
	idTank: "#hladina",
	tank: function(){
		return  paper.select(this.idTank);},
	animateComponentTank: function(fillPerc) {
		if (fillPerc === undefined || fillPerc < 0) {
			fillPerc = 0;
		}
		var perHeight = 600 * (fillPerc / 100);
		var perY = 1912 - perHeight;
		this.tank().animate  (	{
			height: perHeight,
			y: perY
		}, 800);
		return console.log("animacia tanku " + fillPerc);
	}
};
\end{lstlisting}

Vytvorila som objekt Tank medzi jeho atribúty patria: idTank, funkcia tank, a animateComponentTank. IdTank - je stringové - je to id, ktoré som získala zo svg súboru, alebo cez Inkscape ako Label. Funkcia tank - vyberie daný objekt, ktorý chcem ovládať. Pomocou Tank.tank() môžem volať funkcie z Snap knižnice. n 

Zanimovanie tanku je realizované v funkcii animateComponentTank - kde parametrom je v percentách udané o koľko sa ma zdvihnúť hladina nadrze. 
Využívam funkciu animate. Kde v prvom parametri - mením výšku a os y. Hodnotu perHeight je výška 600, ktorú vynásobím percentom o ktoré sa ma posunúť. PerY je hodnota, o ktorú sa posuniem po y-osi. Je vypočitaná ako 1912 co je y prázdnej nádrže a je od nej odpočítaná hodnota výšky. 
Ďalší parameter pri funkcii animate() je rýchlosť animácie vyjadrená v milisekundách.

\subsection{Ventil}

\begin{lstlisting}
var Valve = {
	idValve: "#ventil",
	valve: function (){ return paper.select(this.idValve);},
	colorValve: "red",
	changeIsOpen: function (isOpened) {
		isOpened = (isOpened) ? 0 : 1;
		this.colorValve = (isOpened) ?   "red" : "green";
		this.valve().attr({fill: this.colorValve});
		return;);
	}
}
\end{lstlisting}

Farba sa dá zmeniť aj príkazom 

\begin{verbatim}
Valve.valve().attr({fill: “green”});.
\end{verbatim}

Názov farby môže byť uvedený slovne, alebo ako RGB. 

Zmena farby Valve - \begin{verbatim} this.valve().attr  ({fill: this.colorValve}); \end{verbatim}

