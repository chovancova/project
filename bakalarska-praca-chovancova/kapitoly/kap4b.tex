\chapter{Postup vytvorenia komponentov}

UML diagram... 

Spôsob vytvorenia grafických komponentov je nasledovný. Najprv používateľ vytvorí SVG súbor, následne ho načíta, a vytvorí funkcie v JavaScripte na ovládanie atribútov SVG elementu. 
Alternatívna možnosť je vytvoriť SVG elementy prostredníctvom JavaScriptovej knižnice a nenačítavať súbor. 


\begin{table}[hp]
	
\begin{center}
		\begin{tabular}{|c|c|c|c|}
		\hline \textbf{akcia} & \textbf{SVG akcia} & \textbf{JavaScript akcia} & \textbf{Popis} \\ 
		\hline Animácia &  &  &  \\ 
		\hline Nastavenie farby &  &  &  \\ 
		\hline Transformácia &  &  &  \\ 
		\hline Skryvanie  &  & attr({visibility: true}) &  \\ 
		\hline  &  &  &  \\ 
		\hline  &  &  &  \\ 
		\hline 
	\end{tabular} 
\end{center}
	
	\caption{Mapovacia tabuľka}
	\label{haha}
\end{table}

\section{Použitie SVG v HTML dokumente}

SVG sa dá použiť a vytvoriť viacerými spôsobmi:
\begin{itemize}
	\item priamo v HTML dokumente - inline, 
	\item načítanie z oddeleného SVG súboru,
	\item načítanie pomocou JavaScriptovej knižnice.
\end{itemize}


 \subsection{Vytvorenenie SVG }
 
 \subsubsection{Cez program WYSWING}
 
 \subsubsection{načítanie JavaScriptovú knižnicu}

Postup (načítanie súboru v tele JavaScriptovej metóde): 
\begin{enumerate}
	\item Načítať knižnicu Snap.svg.js do HTML súboru. 
	\item Pridať atribút onPageLoad(); do definicie body.
	\item Pridať HTML tag $<$svg$>$ do tela HTML a nastaviť v ňom požadovanú veľkosť cez viewBox.
	\item Vytvoriť JavaScriptový súbor, alebo tag $<$script$>$, ktorý bude obsahovať funkcie. 
	\item Vytvorenie funkcie na načítanie Snap API, a .SVG súboru. 
	\item V tele funkcie inicializácia Snap Canvasu. To znamená, kde konkrétne v HTML stránke sa zobrazí.
	\begin{itemize}
		\item s = Snap() - najbližšie voľné miesto
		\item s = Snap(šírka, výška) 
		\item s = Snap(HTMLtag) - id tagu $<$svg$>$, ktoré sa pridalo v bode č. 3
	\end{itemize}
	\item Načítanie .SVG súboru cez funkciu Snap.load(), s parametrami: názov súboru a funkcie s parametrom f. 
	\item Zobrazenie súboru cez príkaz s.append(f);, ekvivalenté zápisy: s.appendAll(f);, s.add(f);. 
	\item V HTML stránke sa zobrazí daný .SVG súbor. 
	
	
\end{enumerate}

Postup ovládania SVG elementu:

\begin{enumerate}
	\item Vytvorenie novej funkcie. 
	\begin{itemize}
		\item anonymná funkcia 
		\item pomenovaná funkcia
		\item objekt, v ktorom bude zadefinovaná funkcia. 
	\end{itemize}
	\item Nová premenná var, ktorá obsahuje id SVG elementu, ktorý sa ide ovládať. (Na zistenie id SVG vid Postup krokov na zistenie id.)
	\item Vytvorenie funkcie cez ktorú sa bude pristupovať k API Snap knižnice. 
	\item s.select(id SVG)
	\item V tejto chvíli je možné volať funkcie z Snap API príkazom: funkcia().funkciaAPISnap.. 
	\begin{itemize}
		\item .animate() - animácia
		\item .attr() - nastavenie atribútu
		\item .add() 
		\item TODO
		\
	\end{itemize}
	
	
	
\end{enumerate}

TODO - odkazat tuto na vsetky mozne attributy, ktore sa daju zmeniť.










\section{JavaScript}











