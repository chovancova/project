\chapter{Postup vytvorenia komponentov}

UML diagram... 

Spôsob vytvorenia grafických komponentov je nasledovný. Najprv používateľ vytvorí SVG súbor, následne ho načíta, a vytvorí funkcie v JavaScripte na ovládanie atribútov SVG elementu. 
Alternatívna možnosť je vytvoriť SVG elementy prostredníctvom JavaScriptovej knižnice a nenačítavať súbor. 


\begin{table}[hp]
	
\begin{center}
		\begin{tabular}{|c|c|c|c|}
		\hline \textbf{akcia} & \textbf{SVG akcia} & \textbf{JavaScript akcia} & \textbf{Popis} \\ 
		\hline Animácia &  &  &  \\ 
		\hline Nastavenie farby &  &  &  \\ 
		\hline Transformácia &  &  &  \\ 
		\hline Skryvanie  &  & attr({visibility: true}) &  \\ 
		\hline  &  &  &  \\ 
		\hline  &  &  &  \\ 
		\hline 
	\end{tabular} 
\end{center}
	
	\caption{Mapovacia tabuľka}
	\label{haha}
\end{table}

\section{Vytvorenie SVG v HTML dokumente}

SVG sa dá vytvoriť a zobraziť  týmito spôsobmi:
\begin{itemize}
	\item priamo v HTML dokumente - inline, 
	\item cez nástroj Inkscape,
	\item pomocou JavaScriptovej knižnice.
\end{itemize}


 \subsection{Vytvorenenie SVG }
 
 \subsubsection{Cez program}
 
 \subsubsection{Cez JavaScriptovú knižnicu}















\section{JavaScript}











