\chapter*{Úvod}
\addcontentsline{toc}{chapter}{Úvod}

Témou mojej práce je vizualizácia dát získaných pomocou SCADA systémov. 

Náplňou prvej časti práce je všeobecné popísanie dostupných JavaScriptových knižníc na tvorbu a vizualizáciu grafických komponentov. 

Ďalšia časť sa zaoberá návrhom a tvorbou grafických komponentov. Vzorová sada je vytvorená pomocou programu Inkscape a na vizualizovanie využíva JavaScriptovú knižnicu Snap.svg. Možnosti knižnice sú popísané v práci ako aj spôsob ako je knižnica schopná animovať a meniť atribúty. Je uvedený podrobnejší postup implementácie prečerpávacej stanice, ktorá je zo sady grafických komponentov. Bol navrhnutý jednoduchý interface, pomocou ktorého komponenty komunikujú so serverovou časťou SCADA systému. 

Ďalšia kapitola popisuje v stručnosti návrh REST API. V poslednej kapitole je výkonnostné zhodnotenie JavaScriptových knižníc Raphael a Snap.svg. 

Zdrojové kódy práce sú udržiavané v Git repository.\cite{github} Dostupné na:  \\https://github.com/chovancova/project, ktorý obsahuje odkaz na webovú stránku s implementovanou grafickou sadou komponentov. \\ 


\chapter{Analýza súčasného stavu}

V súčasnosti je internet bežnou súčasťou každodenného života. Doposiaľ bol rozšírený iba na stolných počítačoch, ale nástupom moderných mobilných zariadený sa stal neodmysliteľnou súčasťou života. V dnešnej dobe je požiadavka, aby desktopové programy, ktoré išli spustiť iba cez určitý program, sa dali spustiť aj v mobilných zariadeniach. 

V súčasnosti je v IPESOFT s.r.o. software, ktorý dokáže vizualizovať dáta z technológii pomocou "hrubých klientov",  čo sú natívne (.exe) Windows aplikácie. Aktuálna webová prezentácia takýchto dát nespĺňa súčasné štandardy pre moderné webové aplikácie a preto bolo potrebné nájsť nový spôsob vizualizácie na webe, ktorý bude v budúcnosti použiteľný na rôznych platformách, nielen na PC. 




\chapter{Cieľ práce}
 Cieľ práce je vytvoriť postup ako tvoriť a vizualizovať grafické komponenty. Produktom bakalárskej práce je sada grafických komponentov na vizualizáciu technologických procesov s využitím HTML 5 štandardov. 


\section*{Postup práce}
 
\begin{enumerate}
\item  Analýza požiadaviek, prieskum možnosti využitia \acs{WYSIWYG}\par editorov na tvorbu grafických komponent s možnosťou exportu do formátov \acs{SVG}, \acs{JSON}, \acs{XML}, alebo JavaScript.
\item Výber vhodných open-source knižníc na tvorbu grafických komponent kompatibilných s HTML 5.

\item Popis postupu vizualizácie grafického komponentu. 
\item  Implementácia vzorovej sady grafických komponent.
\item Návrh \acs{REST} \acs{API} na prepojenie grafických komponent so \acs{SCADA} serverom.
\item  Analýza možnosti automatického mapovania API grafických prvkov pomocou metadát na existujúce API dostupné pre SCADA server D2000.

\item  Analýza výkonnosti a výkonnostné obmedzenia.
\end{enumerate}




%pridat / sucastna situacia 
%a ciel prace
%metodika vypracovani


















