\chapter*{Úvod}
\addcontentsline{toc}{chapter}{Úvod}

Téma práce je vizualizácia dát získaných pomocou SCADA systémov. Cieľ práce je nájsť postup tvorby a vizualizácie grafických komponentov. Produktom bakalárskej práce je grafický komponent na vizualizáciu technologických procesov s využitím HTML 5 štandardov. 
V práci je popísaný detailný postup vizualizácie prečerpávacej stanice. Bol navrhnutý jednoduchý interface, pomocou ktorého komponenty komunikujú so serverovou časťou SCADA systému. 

\section*{Súčasná situácia}

V súčasnosti je v IPESOFT s.r.o. software, ktorý dokáže vizualizovať dáta z technológii pomocou "hrubých klientov",  čo sú natívne (.exe) Windows aplikácie. Aktuálna webová prezentácia takýchto dát nespĺňa súčasné štandardy pre moderné webové aplikácie a preto bolo potrebné nájsť nový spôsob vizualizácie na webe, ktorý bude v budúcnosti použiteľný na rôznych platformách, nielen na PC. 


Výsledná webová aplikácia je kompatibilná s štandardmi HTML 5. Riešenie využíva výhradne open-source knižnice s licenciami typu MIT, GNU GPL, BSD, Apache 2. 

Zdrojové kódy práce sú udržiavané v Git repository.\cite{github}\\ 

%Cieľová platforma pre výslednú webovú aplikáciu bude každý webový prehliadač kompatibilný s rodinou štandardov HTML 5. Riešenie bude využívať výhradne open-source knižnice s licenciami typu MIT, GNU GPL, BSD. Zdrojové kódy práce budú udržiavané v Git repository.

%Predbežný postup práce:
%
%\begin{enumerate}
%\item  Analýza požiadaviek, prieskum možnosti využitia \acs{WYSIWYG}\par editorov na tvorbu grafických komponent s možnosťou exportu do formátov \acs{SVG}, \acs{JSON}, \acs{XML}, alebo JavaScript.
%\item Výber vhodných open-source knižníc na tvorbu grafických komponent kompatibilných s HTML 5.
%\item Návrh \acs{REST} \acs{API} na prepojenie grafických komponent so \acs{SCADA} serverom.
%\item  Analýza možnosti automatického mapovania API grafických prvkov pomocou metadát na existujúce API dostupné pre SCADA server D2000.
%\item  Implementácia vzorovej sady grafických komponent.
%\item  Analýza výkonnosti a výkonnostné obmedzenia.
%\end{enumerate}
\newpage
\section*{Postup práce}
 
\begin{enumerate}
\item  Analýza požiadaviek, prieskum možnosti využitia \acs{WYSIWYG}\par editorov na tvorbu grafických komponent s možnosťou exportu do formátov \acs{SVG}, \acs{JSON}, \acs{XML}, alebo JavaScript.
\item Výber vhodných open-source knižníc na tvorbu grafických komponent kompatibilných s HTML 5.

\item Popis postupu vizualizácie grafického komponentu. 
\item  Implementácia vzorovej sady grafických komponent.
\item Návrh \acs{REST} \acs{API} na prepojenie grafických komponent so \acs{SCADA} serverom.
\item  Analýza možnosti automatického mapovania API grafických prvkov pomocou metadát na existujúce API dostupné pre SCADA server D2000.

\item  Analýza výkonnosti a výkonnostné obmedzenia.
\end{enumerate}




%pridat / sucastna situacia 
%a ciel prace
%metodika vypracovani