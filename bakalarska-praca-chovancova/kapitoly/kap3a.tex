\chapter{Knižnica Snap.svg.js}

%CSS selektory ... 

%a podrobnejší popis tejto knižnice.. .

%mapovacia tabuľka.. . (pár kľúč-hodnota atribútov)
%Possible parameters
%
%Please refer to the SVG specification for an explanation of these parameters.
%
%arrow-endstringarrowhead on the end of the path. The format for string is <type>[-<width>[-<length>]]. Possible types: classic, block, open, oval, diamond, none, width: wide, narrow, midium, length: long, short, midium.
%clip-rectstringcomma or space separated values: x, y, width and height
%cursorstringCSS type of the cursor
%cxnumberthe x-axis coordinate of the center of the circle, or ellipse
%cynumberthe y-axis coordinate of the center of the circle, or ellipse
%fillstringcolour, gradient or image
%fill-opacitynumber 
%fontstring 
%font-familystring 
%font-sizenumberfont size in pixels
%font-weightstring 
%heightnumber 
%hrefstringURL, if specified element behaves as hyperlink
%opacitynumber 
%pathstringSVG path string format
%rnumberradius of the circle, ellipse or rounded corner on the rect
%rxnumberhorisontal radius of the ellipse
%rynumbervertical radius of the ellipse
%srcstringimage URL, only works for Element.image element
%strokestringstroke colour
%stroke-dasharraystring[“”, “-”, “.”, “-.”, “-..”, “. ”, “- ”, “--”, “- .”, “--.”, “--..”]
%stroke-linecapstring[“butt”, “square”, “round”]
%stroke-linejoinstring[“bevel”, “round”, “miter”]
%stroke-miterlimitnumber 
%stroke-opacitynumber 
%stroke-widthnumberstroke width in pixels, default is '1'
%targetstringused with href
%textstringcontents of the text element. Use \n for multiline text
%text-anchorstring[“start”, “middle”, “end”], default is “middle”
%titlestringwill create tooltip with a given text
%transformstringsee Element.transform
%widthnumber 
%xnumber 
%ynumber

%\begin{table}
%	\begin{tabular}{|c|c|c|}
%		\hline \rule[-2ex]{0pt}{5.5ex}key  & value & priklad \\ 
%		\hline \rule[-2ex]{0pt}{5.5ex} fill & string farby, RGB & "\#fc0"  \\ 
%		\hline \rule[-2ex]{0pt}{5.5ex} stroke & string farby, RGB  &  \\ 
%		\hline \rule[-2ex]{0pt}{5.5ex}strokeWidth  & 2, // CamelCase &  \\ 
%		\hline \rule[-2ex]{0pt}{5.5ex} "fill-opacity" & 0.5 V ROZSAHU 0-1 &  \\ 
%		\hline \rule[-2ex]{0pt}{5.5ex} width &  &  \\ 
%		\hline \rule[-2ex]{0pt}{5.5ex}  &  &  \\ 
%		\hline \rule[-2ex]{0pt}{5.5ex}  &  &  \\ 
%		\hline \rule[-2ex]{0pt}{5.5ex}  &  &  \\ 
%		\hline \rule[-2ex]{0pt}{5.5ex}  &  &  \\ 
%		\hline 
%	\end{tabular} 
%	\caption{Mapovacia tabulka kluc/hodnota}
%	\label{key}
%\end{table}}

z dokumentácie Snap.svg vybrat zopár príkazov, funkcií - príkladov...


\section{Element.attr(...)}
Vráti alebo nastaví dané atribúty elementu.

Parametre:
\begin{itemize}
	\item objekt - obsahuje pár kľúč-hodnota atribútov, ktoré chcem nastaviť.
	\item string - názov atribútu
\end{itemize}
Vráti buď súčasný element alebo stringovú hodnotu atribútu.

Možné parametre:

\begin{table}
	
	
	
		\caption{Možné parametre Element.atrr(...)}
		\label{parametre:attr}
\end{table}



Použitie:
\begin{lstlisting}
	el.attr({
	fill: "#fc0",
	stroke: "#000",
	strokeWidth: 2, // CamelCase...
	"fill-opacity": 0.5, // or dash-separated names
	width: "*=2" // prefixed values
});
console.log(el.attr("fill")); // #fc0
\end{lstlisting}
