\chapter{Knižnica Snap.svg.js}

%CSS selektory ... 

%a podrobnejší popis tejto knižnice.. .

%mapovacia tabuľka.. . (pár kľúč-hodnota atribútov)
%Possible parameters
%
%Please refer to the SVG specification for an explanation of these parameters.
%
%arrow-endstringarrowhead on the end of the path. The format for string is <type>[-<width>[-<length>]]. Possible types: classic, block, open, oval, diamond, none, width: wide, narrow, midium, length: long, short, midium.
%clip-rectstringcomma or space separated values: x, y, width and height
%cursorstringCSS type of the cursor
%cxnumberthe x-axis coordinate of the center of the circle, or ellipse
%cynumberthe y-axis coordinate of the center of the circle, or ellipse
%fillstringcolour, gradient or image
%fill-opacitynumber 
%fontstring 
%font-familystring 
%font-sizenumberfont size in pixels
%font-weightstring 
%heightnumber 
%hrefstringURL, if specified element behaves as hyperlink
%opacitynumber 
%pathstringSVG path string format
%rnumberradius of the circle, ellipse or rounded corner on the rect
%rxnumberhorisontal radius of the ellipse
%rynumbervertical radius of the ellipse
%srcstringimage URL, only works for Element.image element
%strokestringstroke colour
%stroke-dasharraystring[“”, “-”, “.”, “-.”, “-..”, “. ”, “- ”, “--”, “- .”, “--.”, “--..”]
%stroke-linecapstring[“butt”, “square”, “round”]
%stroke-linejoinstring[“bevel”, “round”, “miter”]
%stroke-miterlimitnumber 
%stroke-opacitynumber 
%stroke-widthnumberstroke width in pixels, default is '1'
%targetstringused with href
%textstringcontents of the text element. Use \n for multiline text
%text-anchorstring[“start”, “middle”, “end”], default is “middle”
%titlestringwill create tooltip with a given text
%transformstringsee Element.transform
%widthnumber 
%xnumber 
%ynumber

%\begin{table}
%	\begin{tabular}{|c|c|c|}
%		\hline \rule[-2ex]{0pt}{5.5ex}key  & value & priklad \\ 
%		\hline \rule[-2ex]{0pt}{5.5ex} fill & string farby, RGB & "\#fc0"  \\ 
%		\hline \rule[-2ex]{0pt}{5.5ex} stroke & string farby, RGB  &  \\ 
%		\hline \rule[-2ex]{0pt}{5.5ex}strokeWidth  & 2, // CamelCase &  \\ 
%		\hline \rule[-2ex]{0pt}{5.5ex} "fill-opacity" & 0.5 V ROZSAHU 0-1 &  \\ 
%		\hline \rule[-2ex]{0pt}{5.5ex} width &  &  \\ 
%		\hline \rule[-2ex]{0pt}{5.5ex}  &  &  \\ 
%		\hline \rule[-2ex]{0pt}{5.5ex}  &  &  \\ 
%		\hline \rule[-2ex]{0pt}{5.5ex}  &  &  \\ 
%		\hline \rule[-2ex]{0pt}{5.5ex}  &  &  \\ 
%		\hline 
%	\end{tabular} 
%	\caption{Mapovacia tabulka kluc/hodnota}
%	\label{key}
%\end{table}}

z dokumentácie Snap.svg vybrat zopár príkazov, funkcií - príkladov...


\section{Element.attr(...)}
Vráti alebo nastaví dané atribúty elementu.

Parametre:
\begin{itemize}
	\item objekt - obsahuje pár kľúč-hodnota atribútov, ktoré chcem nastaviť.
	\item string - názov atribútu
\end{itemize}
Vráti buď súčasný element alebo stringovú hodnotu atribútu.

Možné parametre:

\begin{table}[tp]
	\begin{center}
	
	\begin{tabular}{|l|l|p{4cm}|p{7cm}|}
	\hline \textbf{	Kľúč} & Hodnota & Príklad & Využitie \\ 
		\hline cx & číslo & cx: 50 & x-os súradnica centra kruhu, alebo elipsy \\ 
		\hline cy & číslo & cy: 90 & y-os súradnica centra kruhu, alebo elipsy \\ 
		\hline r & číslo & r: 40 & polomer kruhu, elipsy alebo okruhlých rohov na obdĺžniku \\ 
		\hline rx & číslo & r: 50 & horizontálny polomer elipsy \\ 
		\hline ry & číslo & r: 40 & vertikálny polomer elipsy \\ 
		\hline x & číslo & x: 50 & súradnica x-osi \\ 
		\hline y & číslo & y: 100 & súradnica y-osi \\ 
		\hline width & číslo & width: 500 & šírka \\ 
		\hline height & číslo & height: 10 & výška \\ 
		\hline "fill-opacity" & číslo & "fill-opacity": 0.5 & vyplnenie neprehľadnosti, nadobúda hodnoty od 0 po 1 \\ 
		\hline fill & reťazec & fill: "blue" & vyplnenie farbou, gradientom, obrázkom \\ 
		\hline stroke & reťazec & stroke: "blue" & farba výplne okraja \\ 
		\hline strokeWidth & číslo & strokeWidth: 2 & šírka okraja v pixeloch, predvolené je na 1 \\ 
		\hline strokeLinecap & reťazec & strokeLinecap: "round" & ["butt", "square", "round"] \\ 
		\hline strokeLinejoin & reťazec &  strokeLinejoin: "round" & ["bevel", "round", "miter"] \\ 
		\hline viewBox & pole & & napr. viewBox: [0, 0, 800, 600]
		  \\ 
		\hline strokeDasharray & reťazec &    strokeDasharray: "5 3" & pole čiarok, bodie, pomlčiek \\ 
		\hline font & reťazec &   font: "20px Source Sans Pro, sans-serif", & zmena písma, rodiny písma, veľkosti v pixeloch, a weight \\ 
		\hline transform & reťazec & transform: "t" + [0, 5] +
		"r" + 20 & t - preloží sa na dané súradnice, r - otočí sa o daný úhol udaný v stupňoch \\ 
			\hline path & reťazec & path: "M10,10 210,10" & SVG cesta \\ 
			\hline text & reťazec & text: "snap" & zmení text elementu \\ 
		\hline 
	\end{tabular} 
	
		\end{center}
		\caption{Výber možných parametrov pre funkciu Element.atrr(...)}
		\label{parametre:attr}
\end{table}



Použitie:
\begin{lstlisting}
	el.attr({
	fill: "#fc0",
	stroke: "#000",
	strokeWidth: 2, // CamelCase...
	"fill-opacity": 0.5, // or dash-separated names
	width: "*=2" // prefixed values
});
console.log(el.attr("fill")); // #fc0
\end{lstlisting}
