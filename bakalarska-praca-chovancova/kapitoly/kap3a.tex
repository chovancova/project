\chapter{Knižnica Snap.svg.js}




\section{append()}






\section{Element.attr(...)}
Vráti alebo nastaví dané atribúty elementu.

\subsection{Parametre:}
\begin{itemize}
	\item objekt - obsahuje pár kľúč-hodnota atribútov, ktoré chcem nastaviť.
	\item string - názov atribútu
\end{itemize}
 Niekoľko možných dvojíc parametrov sú v tabuľke \ref{parametre:attr}. Vráti buď súčasný element alebo stringovú hodnotu atribútu.

\subsection{
Použitie:}
\begin{lstlisting}
el.attr({
	fill: "#fc0",
	stroke: "#000",
	strokeWidth: 2, 
});
\end{lstlisting}


\begin{table}[tp]
	\begin{center}
	
	\begin{tabular}{|l|l|p{4cm}|p{6cm}|}
	\hline \textbf{	Kľúč} & \textbf{Hodnota} & \textbf{Príklad použitia} & \textbf{Poznámka} \\ 
		\hline cx & číslo & cx: 50 & x-os súradnica centra kruhu, alebo elipsy \\ 
		\hline cy & číslo & cy: 90 & y-os súradnica centra kruhu, alebo elipsy \\ 
		\hline r & číslo & r: 40 & polomer kruhu, elipsy alebo okruhlých rohov na obdĺžniku \\ 
		\hline rx & číslo & r: 50 & horizontálny polomer elipsy \\ 
		\hline ry & číslo & r: 40 & vertikálny polomer elipsy \\ 
		\hline x, y & číslo & x: 50, y: 100 & súradnica x-osi, y-osi  \\ 
			\hline width, height & číslo & width: 500, height: 10 & šírka, výška \\ 
			\hline "fill-opacity" & číslo & "fill-opacity": 0.5 & vyplnenie neprehľadnosti, nadobúda hodnoty od 0 po 1 \\ 
		\hline fill & reťazec & fill: "blue" & vyplnenie farbou, gradientom, obrázkom \\ 
		\hline stroke & reťazec & stroke: "blue" & farba výplne okraja \\ 
		\hline strokeWidth & číslo & strokeWidth: 2 & šírka okraja v px, default je 1 \\ 
		\hline strokeLinecap & reťazec & strokeLinecap: "round" & ["butt", "square", "round"] \\ 
		\hline strokeLinejoin & reťazec &  strokeLinejoin: "round" & ["bevel", "round", "miter"] \\ 
		\hline viewBox & pole & & napr. viewBox: [0, 0, 800, 600]
		  \\ 
		\hline strokeDasharray & reťazec &    strokeDasharray: "5 3" & pole čiarok, bodie, pomlčiek \\ 
		\hline font & reťazec &   font: "20px Source Sans Pro, sans-serif", & zmena písma, rodiny písma, veľkosti v pixeloch, a weight \\ 
		\hline transform & reťazec & transform: "t" + [0, 5] +
		"r" + 20 & t - preloží sa na dané súradnice, r - otočí sa o daný úhol udaný v stupňoch \\ 
			\hline path & reťazec & path: "M10,10 210,10" & SVG cesta \\ 
			\hline text & reťazec & text: "snap" & zmení text elementu \\ 
		\hline 
	\end{tabular} 
	
		\end{center}
		\caption{Výber možných parametrov pre funkciu Element.atrr(...)}
		\label{parametre:attr}
\end{table}



\section{Element.animate()}




\section{Gradinets, Path String, Colour Parsing}
\subsection{Gradients}
Sú dve možnosti ako zobraziť gradient.
\begin{itemize}
	\item \textbf{Lineárny gradientový formát: } $<$uhol$>$/$<$farba$>$[-$<$farba$>$[:$<$ofset$>$]]*-$<$farba$>$", napríklad: "90 - \#fff - \#000" - $90^o$ gradient z bielej na čiernu
	alebo "0 - \#fff - \#00f : 20 - \#000" $0^o$ gradient z bielej cez modrú pri  20\% a potom na čiernu. 
	\item \textbf{Radial gradient: }"r[($<$fx$>$, $<$fy$>$)]$<$farba$>$[-$<$farba$>$[>$<$ofset$>$]]*-$<$farba$>$", napríklad: "r\#fff-\#000" - je gradient z bielej na čiernu alebo "r(0.25, 0.75)\#fff-\#000" - gradient z bielej na čiernu s zameraním na bod 0.25 a 0.75. Súradnicové body sú zo škály 0...1. Môžu byť len použité pre kruhy, a elipsy.
\end{itemize}

\subsection{Paper.path([pathString])}
Vytvorí $<$path$>$ element podľa daného reťazca. Parameter pozostáva z jedno písmenkových príkazov, nasledovanými bodkami a oddeľovaný argumentami a číslami. 
Napríklad: "M10,20L30,40" - obsahuje príkazy: M s argumentami (10, 20) a L (30, 40). Rozdiel vo veľkosti písma vyjadruje to, či ide o absolútnu, alebo relatívnu cestu. Ak sú malé znaky jedná sa o relatívne, v prípade veľkých znakov absolútna cesta. 
Krátky zoznam príkazov je uvedený v tabuľke ....

\begin{center}
\begin{table}
\begin{center}
	\begin{tabular}{|c|c|c|}
	\hline \textbf{Príkaz} & \textbf{Názov} & \textbf{Parametre} \\
	\hline M & moveto & (x y)+ \\ 
	\hline Z & closepath & (none) \\ 
	\hline L & lineto & (x y)+ \\ 
	\hline H & horizonal lineto & x+ \\ 
	\hline V & vertical lineto & y+ \\ 
	\hline C & curveto & (x1 y1 x2 y2 x y)+ \\ 
	\hline S & smooth curveto & (x2 y2 x y)+ \\ 
	\hline Q & quadratic Bézier curveto & (x1 y1 x y)+ \\ 
	\hline T & smooth quadratic Bézier curveto & (x y)+ \\ 
	\hline 
\end{tabular} 
\end{center}
\caption{Niekoľko príkazov na tvorbu pathString}
\label{prikazyPath}
\end{table}
	\end{center}
		
