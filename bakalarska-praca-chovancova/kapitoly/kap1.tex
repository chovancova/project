\chapter*{Úvod}
\addcontentsline{toc}{chapter}{Úvod}

Téma práce je vizualizácia technologických dát zo SCADA systémov na webe.  Produktom Bakalárskej práce je  vzorová sada grafických komponent na vizualizáciu technologických procesov s využitím HTML 5 štandardov.  Jedná sa o grafické komponenty, ktoré nie sú bežne dostupné na tvorbu interaktívnych webových aplikácií ako napríklad vizualizácie mechanických súčastí hydraulických systémov, alebo technologických liniek, vizualizácie silových a výkonových častí automatizačných sústav. 
Návrh  interface, pomocou ktorého budú tieto komponenty komunikovať so serverovou časťou SCADA systému. 

V súčasnosti je v IPESOFT s.r.o. software, ktorý dokáže vizualizovať dáta z technológii pomocou "hrubých klientov",  čo sú natívne (exe) Windows aplikácie a je technológia,  ktorá dokáže rovnaké dáta zobrazovať na webe. 

Aktuálna webová prezentácia takýchto dát nespĺňa súčasne štandardy pre moderne webové aplikácie a preto je potrebne nájsť nový spôsob vizualizácie na webe, ktorá bude v budúcnosti použiteľná na rôznych platformách, nielen na PC. 


Cieľová platforma pre výslednú webovú aplikáciu bude každý web prehliadač kompatibilný s rodinou štandardov HTML 5. Riešenie bude využívať výhradne open-source knižnice s licenciami typu MIT, GNU GPL, BSD. Zdrojové kódy práce budú udržiavané v Git repository.

Predbežný postup práce:

\begin{enumerate}
\item  Analýza požiadaviek, prieskum možnosti využitia \acs{WYSIWYG}\par editorov na tvorbu grafických komponent s možnosťou exportu do formátov \acs{SVG}, \acs{JSON}, \acs{XML}, alebo JavaScript.
\item Výber vhodných open-source knižníc na tvorbu grafických komponent kompatibilných s HTML 5.
\item Návrh \acs{REST} \acs{API} na prepojenie grafických komponent so \acs{SCADA} serverom.
\item  Analýza možnosti automatického mapovania API grafických prvkov pomocou metadát na existujúce API dostupné pre SCADA server D2000.
\item  Implementácia vzorovej sady grafických komponent.
\item  Analýza výkonnosti a výkonnostné obmedzenia.
\end{enumerate}