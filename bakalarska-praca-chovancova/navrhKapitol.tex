\section{Kapitoly by mali byt nasledovne}
\begin{enumerate}
	\item ciel prace 
	\item metodika prace
	\item definovanie zakladnych pojmov / co je html5, co je scada system, a co je svg vektorova grafika (takmer hotove, len nemam scada systemy definovane), 

	\item ANALYZA javascriptovych kniznic / (toto uz je takmer hotove, este tam pridat jquery )
	\subitem Postup \textbf{tvorby} grafickych komponentov cez Snap...
		\item analyza nastrojov - preco som sa rozhodla pouzit inkscape na tvorbu svg obrazkov, //este tam musim spomenut moznost exportu do formatov... 
		
		\subitem postup na vytvorenie OBRAZKA SVG V INKSCAPE a integracia do snapu  \cite[p.~82-5]{Dawber} 
		\subitem priklad vytvorenia precepavacej stanice v Inkscape (toto je uz takmer hotove) 
	\item Postup \textbf{animacie} uz vytvorenych grafickych elementov 
	\subitem Priklad v kode animovanie Precerpavaciej stanice... (Takmer hotove) toto je vlastne ta implementacia vzorovej sady grafickych komponentov
	\item Navrh REST API na prepojenie grafickych komponentov so SCADA SERVEROM  \textit{toto no nemam / mam iba priklad kodu jednoducheho }
	\subitem Analyza moznosti automatickeho mapovania api grafickych prvkov pomocou metadat na existujuce api dostupne pre scada serverr D2000 \textit{tooto nemam vobec}
	\subitem analyza vykonnosti a vykonnostne obmedzenia - toto iba okrajovo spomeniem - vseobecne preco je lepsie svg, a preco nie je vhodne, v prvej kapitole som pisala rozdiel medzi canvas a svg - istym sposobom to je analyza obmedzeni.. 
	\item zhrnutie
	
\end{enumerate}

