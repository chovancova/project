\chapter{Implementácia komponentov}

Keď už máme nakreslený komponent pomocou Inkscape, tak postupujem ďalej. TODO

Pridáme do HTML  
\begin{lstlisting}
<svg  
	id="svgStanica"  
	viewBox="0 0 750 600" 
	width="40%" 
	height="40%" 
	>
</svg>
\end{lstlisting}


id / toto použijeme pri vykreslení 
viewBox - je atribút, ktorý povoľuje špecifikovať danú množinu grafických. aby to fit a vošlo do kontajnera elementu. Hodnoty atribútov v viewBox sú štyri čísla - min-x, min-y, width a height. 
Width a height je šírka a výška - a je možné ich uviesť aj relatívne v percentách, alebo absolútne v pixloch 

\begin{lstlisting}


<body onload="onPageLoad();">

function onPageLoad() {
	PumpingStation("PumpingStation.svg", "#svgStanica" );
}
\end{lstlisting}
Parametre pre PumpingStation je názov svg súboru, a tag v html.
\begin{lstlisting}
var PumpingStation = function(nazovFileSVG, nameHTMLidSVG) \{
	paper = Snap(nameHTMLidSVG);
	Snap.load(nazovFileSVG, function (f) \{
		paper.append(f);
\});
\};
\end{lstlisting}
paper - bude globálna premenná. Vytvorí plochu na  kreslenie, alebo  wraps existujúci SVG element. Ako parametre môžu byť buď šírka, výška, alebo DOM element. 

Pomocou load načítam vytvorený svg súbor. Na plochu ho zobrazím pomocou príkazu append. 

\subsection{Tank}
Zanimovanie stupania a klesania hladiny nadrze. 
\begin{lstlisting}
var Tank = {
	idTank: "#hladina",
	tank: function(){
		return  paper.select(this.idTank);},
	animateComponentTank: function(fillPerc) {
		if (fillPerc === undefined || fillPerc < 0) {
			fillPerc = 0;
		}
		var perHeight = 600 * (fillPerc / 100);
		var perY = 1912 - perHeight;
		this.tank().animate  (	{
			height: perHeight,
			y: perY
		}, 800);
		return console.log("animacia tanku " + fillPerc);
	}
};
\end{lstlisting}

Vytvorila som objekt Tank medzi jeho atribúty patria: idTank, funkcia tank, a animateComponentTank. IdTank - je stringové - je to id, ktoré som získala zo svg súboru, alebo cez Inkscape ako Label. Funkcia tank - vyberie daný objekt, ktorý chcem ovládať. Pomocou Tank.tank() môžem volať funkcie z Snap knižnice. n 

Zanimovanie tanku je realizované v funkcii animateComponentTank - kde parametrom je v percentách udané o koľko sa ma zdvihnúť hladina nadrze. 
Využívam funkciu animate. Kde v prvom parametri - mením výšku a os y. Hodnotu perHeight je výška 600, ktorú vynásobím percentom o ktoré sa ma posunúť. PerY je hodnota, o ktorú sa posuniem po y-osi. Je vypočitaná ako 1912 co je y prázdnej nádrže a je od nej odpočítaná hodnota výšky. 
Ďalší parameter pri funkcii animate() je rýchlosť animácie vyjadrená v milisekundách.

\subsection{Ventil}

\begin{lstlisting}
var Valve = {
	idValve: "#ventil",
	valve: function (){ return paper.select(this.idValve);},
	colorValve: "red",
	changeIsOpen: function (isOpened) {
		isOpened = (isOpened) ? 0 : 1;
		this.colorValve = (isOpened) ?   "red" : "green";
		this.valve().attr({fill: this.colorValve});
		return;);
	}
}
\end{lstlisting}

Farba sa dá zmeniť aj príkazom 

\begin{verbatim}
Valve.valve().attr({fill: “green”});.
\end{verbatim}

Názov farby môže byť uvedený slovne, alebo ako RGB. 

Zmena farby Valve - \begin{verbatim} this.valve().attr  ({fill: this.colorValve}); \end{verbatim}

