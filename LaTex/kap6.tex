
\chapter{Úvod}

Úvod sa obyčajne píše \textbf{až po napísaní jadra práce} – uvádza sa to, čo je napísané!! Jeho význam spočíva v tom, že poskytuje čitateľovi jasné informácie o riešenom probléme a o dôvodoch, prečo sa ho autor rozhodol riešiť. Obsahuje najmä:
\begin{itemize}
\item  výstižný opis skúmanej problematiky, jej významu, stručnú charakteristiku stavu poznania problematiky, špecifiká skúmanej oblasti,
\end{itemize}
\begin{itemize}
\item  stručné oboznámenie s účelom, autorským zámerom a cieľmi práce, zdôvodnenie jej dôležitosti a významu, prípadne vzťah k iným prácam podobného zamerania,
\end{itemize}
\begin{itemize}
\item  stručný prehľad najpodstatnejších zdrojov, literárnych i iných, využitých pri spracovaní práce.
\end{itemize}


\textbf{\textbf{Tu by bolo dobré zoznámiť a zaradiť problematiku práce. 
}}Je dobré mať na pamäti, na základe akých kriterii bude oponent hodnotit túto prácu. 
Náročnosť zadania sa hodnotí slovne ako  malá, stredná, veľká na základe nasledujúcich kritérii:

\begin{itemize}

\item[$\heartsuit$]{teoretické znalosti,}

\item[$\ast$]{invenčnosť, tvorivosť,}

\item{experimenálna činnosť}

\item{technické práce vrátane programovania,}

\item{návrh algoritmu, datových štruktúr,}

\item{informačno rešeržný prieskum a syntéza.}

\end{itemize}

Dôležitejšie pre záverečné hodnotenie je však je bodové hodnotenie 
na základe nasledujúcich kritérii:

\begin{enumerate}

\item{Hĺbka analýzy vo vzťahu k téme [10b]}

\item[1b)]{Adekvátnosť použitých metód [15b]}

\item{Splnenie cieľov zadania [20b]}

\item{Kvalita riešenia [15b]}

\item{Logická stavba, nadvúznosť, úplnosť, zrozumiteľnosť [10b]}

\item{Formálna gramatická úroveň práce, dokumentácie a prezentácie 10b}

\end{enumerate}

Ak by oponent robil toto hodnotenie v Exelovskej tabuľke a v bunke $D21$ by mal
súčet bodov, potom výsledná známka bude

{\scriptsize
\begin{verbatim}
=if(D21>=90;"A";if(D21>=80;"B";if(D21>=70;"C";if(D21>=60;"D";if(D21>=50;"E";"Fx"))))).
\end{verbatim}
}

\chapter{Nadpis}

Tu je potrebné popísať doteraz získané poznatky z problematiky. 
Nezabudnúť dôsledne citovať autorov článkov, kníh aj internetových publikácií 
napr. monografia \cite{berman}. 
V prameňoch -- spravidla posledná kapitola -- treba uviesť všetku použitú literatúru. 
\textbf{Nemala by obsahovať tie zdroje, ktoré nie sú v práci citované.}
A tiež nie je vhodné citovať nedôveryhodné zdroje ako sú Wikipédia ap.

Jadro práce sa člení na kapitoly a podkapitoly – t.j. podčasti druhej, prípadne tretej úrovne. Vlastný text obsahuje vlastnými slovami formulované úvahy, analýzy, výpočty, vlastné myšlienky – závery a návrhy, parafrázy (vlastnými slovami prerozprávaný originálny text – treba uviesť odkaz na zdroj), výťahy (krátke zhrnutia originálnej pasáže – uviesť odkaz na zdroj), citáty (doslovne prevzatý text – píše sa vždy v úvodzovkách s uvedením odkazu na zdroj), ilustrácie (grafy, obrázky, tabuľky, schémy – u prevzatých je treba vždy uviesť zdroj).

Jadro záverečných prác spravidla obsahuje tieto časti:
\begin{itemize}
\item  charakteristika a súčasný stav riešenej problematiky v SR a vo svete,
\end{itemize}
\begin{itemize}
\item  cieľ záverečnej práce a použité metódy práce (charakteristika použitých metód, techník a postupov, ich využitie),
\end{itemize}
\begin{itemize}
\item  výsledky práce a diskusia.
\end{itemize}

Jadro bakalárskej a diplomovej práce je vhodné rozdeliť na 2 - 4 kapitoly (podľa potreby a dohody s vedúcim - ich počet nie je direktívne stanovený). Vyššie uvedené časti môžu byť samostatnými kapitolami, alebo sa môžu spájať, napr. cieľ a metódy alebo výsledky a diskusia. Tie môžu byť ďalej členené na podkapitoly druhej, výnimočne tretej úrovne. Názvy jednotlivých kapitol a podkapitol musia vystihovať podstatu ich obsahu.

\section{Obsah práce}

b) \textbf{Obsah a štruktúra jadra praktickej}, resp. prípadovej štúdie:
\begin{itemize}
\item  Charakteristika riešeného problému a opis najdôležitejších poznatkov z odbornej a vedeckej literatúry, ktoré sa vzťahujú k danému problému, či prípadu (cca 10 – 15 strán).
\end{itemize}
\begin{itemize}
\item  Ciele a metódy práce – presná a výstižná charakteristika riešenej problematiky a zdôvodnenie zvoleného postupu a použitých metód (cca 1 – 3 strany).
\end{itemize}
\begin{itemize}
\item  Výsledky práce predstavujú výsledky skúmania objektu, ktorého sa téma týka a výsledky analýzy konkrétneho skúmaného prípadu.
\begin{itemize}
  \item \textbf{Analýzou objektu} (organizácie, podniku, odvetvia, javu, procesu, dokumentácie a i.) rozumieme proces identifikácie, opisu, rozboru prvkov, funkcií a vzťahov objektu pomocou zvoleného postupu. 
  \end{itemize}
\begin{itemize}
  \item \textbf{Analýza prípadu }predstavuje systematický, primerane podrobný a výstižný opis konkrétneho prípadu a jeho súvislostí z praxe krízového manažmentu (bezpečnostného manažmentu, riadenia záchranných prác a pod.), identifikácia a interpretácia podstatných zistení (cca 10 strán).
\end{itemize}
\end{itemize}
\begin{itemize}
\item Diskusia má v tomto prípade podobu zhrnutia a návrhov opatrení, t.j. obsahuje zhrnutie pozitívnych a negatívnych poznatkov o objekte, procese či prípade, porovnanie s inými podobnými prípadmi a poznatkami z preštudovanej literatúry, spracovanie súboru opatrení alebo uceleného návrhu na zvýšenie účinnosti, na zefektívnenie činností a operácií, či na odstránenie problémových miest (5 – 10 strán).
\end{itemize}

\section{Ďalšia forma}

V prípade realizácie vlastného prieskumu (sociologického, ekonomického a pod.) je nevyhnutné vypracovať jeho projekt. V ňom je potrebné vymedziť:
\begin{itemize}
\item  cieľ, pracovné hypotézy, úlohy prieskumu,
\end{itemize}
\begin{itemize}
\item  použité metódy a techniky prieskumu,
\end{itemize}
\begin{itemize}
\item  organizáciu a priebeh vykonania prieskumu,
\end{itemize}
\begin{itemize}
\item  popis prieskumnej vzorky (objektu),
\end{itemize}
Podľa projektu realizovať plánované etapy prieskumu, ktorými sú najmä:
\begin{itemize}
\item  zber materiálov (faktov) pomocou metód a techník výskumu,
\end{itemize}
\begin{itemize}
\item  spracovanie informácií, ich výber a členenie na jednotlivé zložky, kvantifikácia získaných dát, ich matematicko-štatistické spracovanie, grafické znázornenie,
\end{itemize}
\begin{itemize}
\item  prehľad výsledkov prieskumu - interpretácia získaných údajov, predstavujúca kvalitatívne hodnotenie materiálu s formuláciou čiastkových a celkových záverov, doložených logickou argumentáciou a dokumentovaných spracovaným materiálom z prieskumu a vyvodenie teoretických záverov i odporúčaní pre prax,
\end{itemize}
\begin{itemize}
\item celkové spracovanie správy o priebehu a výsledkoch prieskumu je súčasťou záverečnej práce. Interpretácii výsledkov prieskumu je možno venovať jednu kapitolu, alebo jej časť.
\end{itemize}


\section{5 CITOVANIE A ZOZNAM BIBLIOGRAFICKÝCH ODKAZOV}
Citovanie a citáty slúžia na argumentáciu. Vhodné sú najmä tam, kde svojou výstižnosťou ušetria zdĺhavý výklad a v prípade, keď vlastné názory a argumenty možno podoprieť citátom autority v odbore. Osamotený citát bez vlastnej argumentácie nemá platnosť vecného dôkazu. Citáty využívame aj na prezentáciu názorov, s ktorými polemizujeme, alebo na porovnanie so zistenými faktami alebo inými názormi.

Pri používaní citátov treba rešpektovať nasledujúce pravidlá:
\begin{itemize}
\item  citát musí byť uvedený doslovne a presne so všetkými typografickými zvláštnosťami (vrátane možných chýb - preklepov) a vždy v úvodzovkách,
\end{itemize}
\begin{itemize}
\item  citát treba vybrať tak, aby vyjadroval ucelenú myšlienku a nebol po vyňatí z pôvodného kontextu porušený jeho zmysel,
\end{itemize}
\begin{itemize}
\item  prevzaté citáty overiť v origináli (ak je dostupný), u neoverených citátov uviesť zdroj, z ktorého boli prevzaté,
\end{itemize}
\begin{itemize}
\item  citáty treba používať priamo v texte záverečnej práce (nie pod čiarou),
\end{itemize}
\begin{itemize}
\item  odlišujeme citát (doslovný text) od jeho výkladu vlastnými slovami. Pri použití voľného výkladu myšlienky autora (jej parafrázy) nedávame text do úvodzoviek, ale označíme ho uvedením pôvodného zdroja podľa niektorého z nižšie uvedených príkladov (podľa STN ISO 690 a STN ISO 690-2).
\end{itemize}
Technikou citovania rozumieme spájanie miesta v texte so záznamami o dokumentoch, ktoré sú v zozname bibliografických odkazov pri rešpektovaní medzinárodných noriem ISO (Kucianová, 2008).1


\chapter{Záver}
Záver musí byť vyústením výkladu, úvah a argumentov uvedených v jadre práce. Mal by obsahovať najmä:
\begin{itemize}
\item  vecné zhrnutie hlavných myšlienok a dosiahnutých výsledkov,
\end{itemize}
\begin{itemize}
\item  vyjadrenie a sumarizáciu vlastného pohľadu na riešený problém a vlastného prínosu,
\end{itemize}
\begin{itemize}
\item  zhodnotenie splnenia cieľa záverečnej práce (overenia pracovných hypotéz),
\end{itemize}
\begin{itemize}
\item  naznačenie sporných otázok, otázok do diskusie, otvorených problémov,
\end{itemize}
\begin{itemize}
\item  formuláciu praktických odporúčaní nadväzujúcich na výklad, analýzy a argumenty obsiahnuté v jadre práce.
\end{itemize}